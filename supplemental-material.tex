\PassOptionsToPackage{unicode=true}{hyperref} % options for packages loaded elsewhere
\PassOptionsToPackage{hyphens}{url}
%
\documentclass[]{book}
\usepackage{lmodern}
\usepackage{amssymb,amsmath}
\usepackage{ifxetex,ifluatex}
\usepackage{fixltx2e} % provides \textsubscript
\ifnum 0\ifxetex 1\fi\ifluatex 1\fi=0 % if pdftex
  \usepackage[T1]{fontenc}
  \usepackage[utf8]{inputenc}
  \usepackage{textcomp} % provides euro and other symbols
\else % if luatex or xelatex
  \usepackage{unicode-math}
  \defaultfontfeatures{Ligatures=TeX,Scale=MatchLowercase}
\fi
% use upquote if available, for straight quotes in verbatim environments
\IfFileExists{upquote.sty}{\usepackage{upquote}}{}
% use microtype if available
\IfFileExists{microtype.sty}{%
\usepackage[]{microtype}
\UseMicrotypeSet[protrusion]{basicmath} % disable protrusion for tt fonts
}{}
\IfFileExists{parskip.sty}{%
\usepackage{parskip}
}{% else
\setlength{\parindent}{0pt}
\setlength{\parskip}{6pt plus 2pt minus 1pt}
}
\usepackage{hyperref}
\hypersetup{
            pdftitle={Supplemental Material for An Exploration of Exploration: Measuring the ability of lexicase selection to find obscure pathways},
            pdfauthor={Jose Guadalupe Hernandez, Alexander Lalejini, and Charles Ofria},
            pdfborder={0 0 0},
            breaklinks=true}
\urlstyle{same}  % don't use monospace font for urls
\usepackage{color}
\usepackage{fancyvrb}
\newcommand{\VerbBar}{|}
\newcommand{\VERB}{\Verb[commandchars=\\\{\}]}
\DefineVerbatimEnvironment{Highlighting}{Verbatim}{commandchars=\\\{\}}
% Add ',fontsize=\small' for more characters per line
\usepackage{framed}
\definecolor{shadecolor}{RGB}{248,248,248}
\newenvironment{Shaded}{\begin{snugshade}}{\end{snugshade}}
\newcommand{\AlertTok}[1]{\textcolor[rgb]{0.94,0.16,0.16}{#1}}
\newcommand{\AnnotationTok}[1]{\textcolor[rgb]{0.56,0.35,0.01}{\textbf{\textit{#1}}}}
\newcommand{\AttributeTok}[1]{\textcolor[rgb]{0.77,0.63,0.00}{#1}}
\newcommand{\BaseNTok}[1]{\textcolor[rgb]{0.00,0.00,0.81}{#1}}
\newcommand{\BuiltInTok}[1]{#1}
\newcommand{\CharTok}[1]{\textcolor[rgb]{0.31,0.60,0.02}{#1}}
\newcommand{\CommentTok}[1]{\textcolor[rgb]{0.56,0.35,0.01}{\textit{#1}}}
\newcommand{\CommentVarTok}[1]{\textcolor[rgb]{0.56,0.35,0.01}{\textbf{\textit{#1}}}}
\newcommand{\ConstantTok}[1]{\textcolor[rgb]{0.00,0.00,0.00}{#1}}
\newcommand{\ControlFlowTok}[1]{\textcolor[rgb]{0.13,0.29,0.53}{\textbf{#1}}}
\newcommand{\DataTypeTok}[1]{\textcolor[rgb]{0.13,0.29,0.53}{#1}}
\newcommand{\DecValTok}[1]{\textcolor[rgb]{0.00,0.00,0.81}{#1}}
\newcommand{\DocumentationTok}[1]{\textcolor[rgb]{0.56,0.35,0.01}{\textbf{\textit{#1}}}}
\newcommand{\ErrorTok}[1]{\textcolor[rgb]{0.64,0.00,0.00}{\textbf{#1}}}
\newcommand{\ExtensionTok}[1]{#1}
\newcommand{\FloatTok}[1]{\textcolor[rgb]{0.00,0.00,0.81}{#1}}
\newcommand{\FunctionTok}[1]{\textcolor[rgb]{0.00,0.00,0.00}{#1}}
\newcommand{\ImportTok}[1]{#1}
\newcommand{\InformationTok}[1]{\textcolor[rgb]{0.56,0.35,0.01}{\textbf{\textit{#1}}}}
\newcommand{\KeywordTok}[1]{\textcolor[rgb]{0.13,0.29,0.53}{\textbf{#1}}}
\newcommand{\NormalTok}[1]{#1}
\newcommand{\OperatorTok}[1]{\textcolor[rgb]{0.81,0.36,0.00}{\textbf{#1}}}
\newcommand{\OtherTok}[1]{\textcolor[rgb]{0.56,0.35,0.01}{#1}}
\newcommand{\PreprocessorTok}[1]{\textcolor[rgb]{0.56,0.35,0.01}{\textit{#1}}}
\newcommand{\RegionMarkerTok}[1]{#1}
\newcommand{\SpecialCharTok}[1]{\textcolor[rgb]{0.00,0.00,0.00}{#1}}
\newcommand{\SpecialStringTok}[1]{\textcolor[rgb]{0.31,0.60,0.02}{#1}}
\newcommand{\StringTok}[1]{\textcolor[rgb]{0.31,0.60,0.02}{#1}}
\newcommand{\VariableTok}[1]{\textcolor[rgb]{0.00,0.00,0.00}{#1}}
\newcommand{\VerbatimStringTok}[1]{\textcolor[rgb]{0.31,0.60,0.02}{#1}}
\newcommand{\WarningTok}[1]{\textcolor[rgb]{0.56,0.35,0.01}{\textbf{\textit{#1}}}}
\usepackage{longtable,booktabs}
% Fix footnotes in tables (requires footnote package)
\IfFileExists{footnote.sty}{\usepackage{footnote}\makesavenoteenv{longtable}}{}
\usepackage{graphicx,grffile}
\makeatletter
\def\maxwidth{\ifdim\Gin@nat@width>\linewidth\linewidth\else\Gin@nat@width\fi}
\def\maxheight{\ifdim\Gin@nat@height>\textheight\textheight\else\Gin@nat@height\fi}
\makeatother
% Scale images if necessary, so that they will not overflow the page
% margins by default, and it is still possible to overwrite the defaults
% using explicit options in \includegraphics[width, height, ...]{}
\setkeys{Gin}{width=\maxwidth,height=\maxheight,keepaspectratio}
\setlength{\emergencystretch}{3em}  % prevent overfull lines
\providecommand{\tightlist}{%
  \setlength{\itemsep}{0pt}\setlength{\parskip}{0pt}}
\setcounter{secnumdepth}{5}
% Redefines (sub)paragraphs to behave more like sections
\ifx\paragraph\undefined\else
\let\oldparagraph\paragraph
\renewcommand{\paragraph}[1]{\oldparagraph{#1}\mbox{}}
\fi
\ifx\subparagraph\undefined\else
\let\oldsubparagraph\subparagraph
\renewcommand{\subparagraph}[1]{\oldsubparagraph{#1}\mbox{}}
\fi

% set default figure placement to htbp
\makeatletter
\def\fps@figure{htbp}
\makeatother

\usepackage[]{natbib}
\bibliographystyle{apalike}

\title{Supplemental Material for An Exploration of Exploration: Measuring the ability of lexicase selection to find obscure pathways}
\author{Jose Guadalupe Hernandez, Alexander Lalejini, and Charles Ofria}
\date{2021-06-16}

\begin{document}
\maketitle

{
\setcounter{tocdepth}{1}
\tableofcontents
}
\hypertarget{introduction}{%
\chapter{Introduction}\label{introduction}}

This is the supplemental material associated with our 2021 GPTP contribution entitled, \emph{An Exploration of Exploration: Measuring the ability of lexicase selection to find obscure pathways}.
Preprint forthcoming.

\hypertarget{about-our-supplemental-material}{%
\section{About our supplemental material}\label{about-our-supplemental-material}}

This supplemental material is hosted on \href{https://github.com}{GitHub} using GitHub pages.
The source code and configuration files used to generate this supplemental material can be found in \href{https://github.com/jgh9094/GPTP-2021-Exploration-Of-Exploration}{this GitHub repository}.
We compiled our data analyses and supplemental documentation into this nifty web-accessible book using \href{https://bookdown.org/}{bookdown}.

Our supplemental material includes the following:

\begin{itemize}
\tightlist
\item
  TODO
\end{itemize}

\hypertarget{contributing-authors}{%
\section{Contributing authors}\label{contributing-authors}}

\begin{itemize}
\tightlist
\item
  \href{https://jgh9094.github.io/}{Jose Guadalupe Hernandez}
\item
  \href{https://lalejini.com}{Alexander Lalejini}
\item
  \href{http://ofria.com}{Charles Ofria}
\end{itemize}

\hypertarget{research-overview}{%
\section{Research overview}\label{research-overview}}

Abstract:

\begin{quote}
TODO
\end{quote}

\hypertarget{data-availability}{%
\chapter{Data Availability}\label{data-availability}}

\hypertarget{source-code}{%
\section{Source code}\label{source-code}}

The source code for this work is available on GitHub at \url{https://github.com/jgh9094/GPTP-2021-Exploration-Of-Exploration}.

\hypertarget{experimental-results}{%
\section{Experimental results}\label{experimental-results}}

The data from our experiments are available online in an OSF repository \citep{osf_data} at \url{https://osf.io/xpjft/}.

\hypertarget{compile-and-run-experiments}{%
\chapter{Compile and run experiments}\label{compile-and-run-experiments}}

Here, we provide a guide to compiling and running our experiments using our Docker image.

Please file an \href{https://github.com/jgh9094/GPTP-2021-Exploration-Of-Exploration/issues}{issue on GitHub} if something is unclear or does not work.

\hypertarget{docker}{%
\section{Docker}\label{docker}}

TODO

\hypertarget{getting-the-right-image}{%
\subsection{Getting the right image}\label{getting-the-right-image}}

\hypertarget{dockerhub}{%
\subsubsection{DockerHub}\label{dockerhub}}

\hypertarget{local-build}{%
\subsubsection{Local build}\label{local-build}}

\hypertarget{spinning-up-a-container}{%
\subsection{Spinning up a container}\label{spinning-up-a-container}}

\hypertarget{running-inside-the-container}{%
\subsection{Running inside the container}\label{running-inside-the-container}}

\hypertarget{copying-content-from-the-container}{%
\subsection{Copying content from the container}\label{copying-content-from-the-container}}

\hypertarget{diagnostic-cardinality}{%
\chapter{Diagnostic cardinality}\label{diagnostic-cardinality}}

\hypertarget{overview}{%
\section{Overview}\label{overview}}

\begin{Shaded}
\begin{Highlighting}[]
\CommentTok{# Relative location of data.}
\NormalTok{working_directory <-}
\StringTok{  "experiments/2021-05-27-cardinality/analysis/"}
\CommentTok{# working_directory <- "./"}

\CommentTok{# Settings for visualization}
\NormalTok{cb_palette <-}\StringTok{ "Set2"}
\CommentTok{# Create directory to dump plots}
\KeywordTok{dir.create}\NormalTok{(}\KeywordTok{paste0}\NormalTok{(working_directory, }\StringTok{"imgs"}\NormalTok{), }\DataTypeTok{showWarnings=}\OtherTok{FALSE}\NormalTok{)}
\end{Highlighting}
\end{Shaded}

\hypertarget{analysis-dependencies}{%
\section{Analysis dependencies}\label{analysis-dependencies}}

\begin{Shaded}
\begin{Highlighting}[]
\KeywordTok{library}\NormalTok{(ggplot2)}
\KeywordTok{library}\NormalTok{(tidyverse)}
\KeywordTok{library}\NormalTok{(cowplot)}
\KeywordTok{library}\NormalTok{(viridis)}
\KeywordTok{library}\NormalTok{(RColorBrewer)}
\KeywordTok{source}\NormalTok{(}\StringTok{"https://gist.githubusercontent.com/benmarwick/2a1bb0133ff568cbe28d/raw/fb53bd97121f7f9ce947837ef1a4c65a73bffb3f/geom_flat_violin.R"}\NormalTok{)}
\end{Highlighting}
\end{Shaded}

These analyses were conducted in the following computing environment:

\begin{Shaded}
\begin{Highlighting}[]
\KeywordTok{print}\NormalTok{(version)}
\end{Highlighting}
\end{Shaded}

\begin{verbatim}
##                _                           
## platform       x86_64-pc-linux-gnu         
## arch           x86_64                      
## os             linux-gnu                   
## system         x86_64, linux-gnu           
## status                                     
## major          4                           
## minor          1.0                         
## year           2021                        
## month          05                          
## day            18                          
## svn rev        80317                       
## language       R                           
## version.string R version 4.1.0 (2021-05-18)
## nickname       Camp Pontanezen
\end{verbatim}

\hypertarget{setup}{%
\section{Setup}\label{setup}}

\begin{Shaded}
\begin{Highlighting}[]
\NormalTok{data_loc <-}\StringTok{ }\KeywordTok{paste0}\NormalTok{(}
\NormalTok{  working_directory,}
  \StringTok{"data/timeseries-res-1000g.csv"}
\NormalTok{)}
\NormalTok{data <-}\StringTok{ }\KeywordTok{read.csv}\NormalTok{(}
\NormalTok{  data_loc,}
  \DataTypeTok{na.strings=}\StringTok{"NONE"}
\NormalTok{)}

\NormalTok{data}\OperatorTok{$}\NormalTok{cardinality <-}\StringTok{ }\KeywordTok{as.factor}\NormalTok{(}
\NormalTok{  data}\OperatorTok{$}\NormalTok{OBJECTIVE_CNT}
\NormalTok{)}
\NormalTok{data}\OperatorTok{$}\NormalTok{selection_name <-}\StringTok{ }\KeywordTok{as.factor}\NormalTok{(}
\NormalTok{  data}\OperatorTok{$}\NormalTok{selection_name}
\NormalTok{)}

\NormalTok{data}\OperatorTok{$}\NormalTok{elite_trait_avg <-}
\StringTok{  }\NormalTok{data}\OperatorTok{$}\NormalTok{ele_agg_per }\OperatorTok{/}\StringTok{ }\NormalTok{data}\OperatorTok{$}\NormalTok{OBJECTIVE_CNT}

\NormalTok{data}\OperatorTok{$}\NormalTok{unique_start_positions_coverage <-}
\StringTok{  }\NormalTok{data}\OperatorTok{$}\NormalTok{uni_str_pos }\OperatorTok{/}\StringTok{ }\NormalTok{data}\OperatorTok{$}\NormalTok{OBJECTIVE_CNT}

\CommentTok{####### misc #######}
\CommentTok{# Configure our default graphing theme}
\KeywordTok{theme_set}\NormalTok{(}\KeywordTok{theme_cowplot}\NormalTok{())}
\end{Highlighting}
\end{Shaded}

\hypertarget{exploration-diagnostic-performance}{%
\section{Exploration diagnostic performance}\label{exploration-diagnostic-performance}}

First, we look at performance over time.
Specifically, we look at the normalized aggregage score of the most performant individuals over time.
To control for different cardinalities having different maximum scores, we normalized performances (by dividing by cardinality) to values between 0 and 100.

\begin{Shaded}
\begin{Highlighting}[]
\NormalTok{elite_trait_ave_fit <-}\StringTok{ }\KeywordTok{ggplot}\NormalTok{(}
\NormalTok{    data,}
    \KeywordTok{aes}\NormalTok{(}
      \DataTypeTok{x=}\NormalTok{gen,}
      \DataTypeTok{y=}\NormalTok{elite_trait_avg,}
      \DataTypeTok{color=}\NormalTok{cardinality,}
      \DataTypeTok{fill=}\NormalTok{cardinality}
\NormalTok{    )}
\NormalTok{  ) }\OperatorTok{+}
\StringTok{  }\KeywordTok{stat_summary}\NormalTok{(}\DataTypeTok{geom=}\StringTok{"line"}\NormalTok{, }\DataTypeTok{fun=}\NormalTok{mean) }\OperatorTok{+}
\StringTok{  }\KeywordTok{stat_summary}\NormalTok{(}
    \DataTypeTok{geom=}\StringTok{"ribbon"}\NormalTok{,}
    \DataTypeTok{fun.data=}\StringTok{"mean_cl_boot"}\NormalTok{,}
    \DataTypeTok{fun.args=}\KeywordTok{list}\NormalTok{(}\DataTypeTok{conf.int=}\FloatTok{0.95}\NormalTok{),}
    \DataTypeTok{alpha=}\FloatTok{0.2}\NormalTok{,}
    \DataTypeTok{linetype=}\DecValTok{0}
\NormalTok{  ) }\OperatorTok{+}
\StringTok{  }\KeywordTok{scale_y_continuous}\NormalTok{(}
    \DataTypeTok{name=}\StringTok{"Average trait performance"}\NormalTok{,}
    \DataTypeTok{limits=}\KeywordTok{c}\NormalTok{(}\DecValTok{0}\NormalTok{, }\DecValTok{100}\NormalTok{)}
\NormalTok{  ) }\OperatorTok{+}
\StringTok{  }\KeywordTok{scale_x_continuous}\NormalTok{(}
    \DataTypeTok{name=}\StringTok{"Generation"}
\NormalTok{  ) }\OperatorTok{+}
\StringTok{  }\KeywordTok{scale_fill_brewer}\NormalTok{(}
    \DataTypeTok{name=}\StringTok{"Cardinaltiy"}\NormalTok{,}
    \DataTypeTok{palette=}\NormalTok{cb_palette}
\NormalTok{  ) }\OperatorTok{+}
\StringTok{  }\KeywordTok{scale_color_brewer}\NormalTok{(}
    \DataTypeTok{name=}\StringTok{"Cardinaltiy"}\NormalTok{,}
    \DataTypeTok{palette=}\NormalTok{cb_palette}
\NormalTok{  )}
\NormalTok{elite_trait_ave_fit}
\end{Highlighting}
\end{Shaded}

\includegraphics{supplemental-material_files/figure-latex/unnamed-chunk-5-1.pdf}

\hypertarget{final-performance}{%
\subsection{Final performance}\label{final-performance}}

Next, we look only at the final performances of each treatment

\begin{Shaded}
\begin{Highlighting}[]
\NormalTok{final_data <-}\StringTok{ }\KeywordTok{filter}\NormalTok{(data, gen}\OperatorTok{==}\KeywordTok{max}\NormalTok{(data}\OperatorTok{$}\NormalTok{gen))}
\NormalTok{elite_trait_ave_fit_final <-}\StringTok{ }\KeywordTok{ggplot}\NormalTok{(}
\NormalTok{    final_data,}
    \KeywordTok{aes}\NormalTok{(}\DataTypeTok{x=}\NormalTok{cardinality, }\DataTypeTok{y=}\NormalTok{elite_trait_avg, }\DataTypeTok{fill=}\NormalTok{cardinality)}
\NormalTok{  ) }\OperatorTok{+}
\StringTok{  }\KeywordTok{geom_flat_violin}\NormalTok{(}
    \DataTypeTok{position =} \KeywordTok{position_nudge}\NormalTok{(}\DataTypeTok{x =} \FloatTok{.2}\NormalTok{, }\DataTypeTok{y =} \DecValTok{0}\NormalTok{),}
    \DataTypeTok{alpha =} \FloatTok{.8}\NormalTok{,}
    \DataTypeTok{scale=}\StringTok{"width"}
\NormalTok{  ) }\OperatorTok{+}
\StringTok{  }\KeywordTok{geom_point}\NormalTok{(}
    \DataTypeTok{mapping=}\KeywordTok{aes}\NormalTok{(}\DataTypeTok{color=}\NormalTok{cardinality),}
    \DataTypeTok{position =} \KeywordTok{position_jitter}\NormalTok{(}\DataTypeTok{width =} \FloatTok{.15}\NormalTok{),}
    \DataTypeTok{size =} \FloatTok{.5}\NormalTok{,}
    \DataTypeTok{alpha =} \FloatTok{0.8}
\NormalTok{  ) }\OperatorTok{+}
\StringTok{  }\KeywordTok{geom_boxplot}\NormalTok{(}
    \DataTypeTok{width =} \FloatTok{.1}\NormalTok{,}
    \DataTypeTok{outlier.shape =} \OtherTok{NA}\NormalTok{,}
    \DataTypeTok{alpha =} \FloatTok{0.5}
\NormalTok{  ) }\OperatorTok{+}
\StringTok{  }\KeywordTok{scale_y_continuous}\NormalTok{(}
    \DataTypeTok{name=}\StringTok{"Average trait performance"}\NormalTok{,}
    \DataTypeTok{limits=}\KeywordTok{c}\NormalTok{(}\DecValTok{0}\NormalTok{, }\DecValTok{100}\NormalTok{)}
\NormalTok{  ) }\OperatorTok{+}
\StringTok{  }\KeywordTok{scale_x_discrete}\NormalTok{(}
    \DataTypeTok{name=}\StringTok{"Cardinality"}
\NormalTok{  ) }\OperatorTok{+}
\StringTok{  }\KeywordTok{scale_fill_brewer}\NormalTok{(}
    \DataTypeTok{name=}\StringTok{"Cardinaltiy"}\NormalTok{,}
    \DataTypeTok{palette=}\NormalTok{cb_palette}
\NormalTok{  ) }\OperatorTok{+}
\StringTok{  }\KeywordTok{scale_color_brewer}\NormalTok{(}
    \DataTypeTok{name=}\StringTok{"Cardinaltiy"}\NormalTok{,}
    \DataTypeTok{palette=}\NormalTok{cb_palette}
\NormalTok{  ) }\OperatorTok{+}
\StringTok{  }\KeywordTok{theme}\NormalTok{(}
    \DataTypeTok{legend.position=}\StringTok{"none"}
\NormalTok{  )}
\NormalTok{elite_trait_ave_fit_final}
\end{Highlighting}
\end{Shaded}

\includegraphics{supplemental-material_files/figure-latex/unnamed-chunk-6-1.pdf}

\hypertarget{unique-starting-positions}{%
\section{Unique starting positions}\label{unique-starting-positions}}

Next, we analyze the number of unique starting position maintained by populations.

\begin{Shaded}
\begin{Highlighting}[]
\KeywordTok{ggplot}\NormalTok{(data, }\KeywordTok{aes}\NormalTok{(}\DataTypeTok{x=}\NormalTok{gen, }\DataTypeTok{y=}\NormalTok{uni_str_pos, }\DataTypeTok{color=}\NormalTok{cardinality)) }\OperatorTok{+}
\StringTok{  }\KeywordTok{stat_summary}\NormalTok{(}\DataTypeTok{geom=}\StringTok{"line"}\NormalTok{, }\DataTypeTok{fun=}\NormalTok{mean) }\OperatorTok{+}
\StringTok{  }\KeywordTok{stat_summary}\NormalTok{(}
    \DataTypeTok{geom=}\StringTok{"ribbon"}\NormalTok{,}
    \DataTypeTok{fun.data=}\StringTok{"mean_cl_boot"}\NormalTok{,}
    \DataTypeTok{fun.args=}\KeywordTok{list}\NormalTok{(}\DataTypeTok{conf.int=}\FloatTok{0.95}\NormalTok{),}
    \DataTypeTok{alpha=}\FloatTok{0.2}\NormalTok{,}
    \DataTypeTok{linetype=}\DecValTok{0}
\NormalTok{  ) }\OperatorTok{+}
\StringTok{  }\KeywordTok{scale_y_continuous}\NormalTok{(}
    \DataTypeTok{name=}\StringTok{"Unique starting positions (population)"}\NormalTok{,}
\NormalTok{  ) }\OperatorTok{+}
\StringTok{  }\KeywordTok{scale_x_continuous}\NormalTok{(}
    \DataTypeTok{name=}\StringTok{"Generation"}
\NormalTok{  )}
\end{Highlighting}
\end{Shaded}

\includegraphics{supplemental-material_files/figure-latex/unnamed-chunk-7-1.pdf}

Different cardinalities have numbers of possible starting positions, so next, we look at the proportion of starting positions (out of all possible) maintained by populations.

\begin{Shaded}
\begin{Highlighting}[]
\NormalTok{unique_start_positions_coverage_fig <-}\StringTok{ }\KeywordTok{ggplot}\NormalTok{(}
\NormalTok{    data,}
    \KeywordTok{aes}\NormalTok{(}
      \DataTypeTok{x=}\NormalTok{gen,}
      \DataTypeTok{y=}\NormalTok{unique_start_positions_coverage,}
      \DataTypeTok{color=}\NormalTok{cardinality,}
      \DataTypeTok{fill=}\NormalTok{cardinality}
\NormalTok{    )}
\NormalTok{  ) }\OperatorTok{+}
\StringTok{  }\KeywordTok{stat_summary}\NormalTok{(}\DataTypeTok{geom=}\StringTok{"line"}\NormalTok{, }\DataTypeTok{fun=}\NormalTok{mean) }\OperatorTok{+}
\StringTok{  }\KeywordTok{stat_summary}\NormalTok{(}
    \DataTypeTok{geom=}\StringTok{"ribbon"}\NormalTok{,}
    \DataTypeTok{fun.data=}\StringTok{"mean_cl_boot"}\NormalTok{,}
    \DataTypeTok{fun.args=}\KeywordTok{list}\NormalTok{(}\DataTypeTok{conf.int=}\FloatTok{0.95}\NormalTok{),}
    \DataTypeTok{alpha=}\FloatTok{0.2}\NormalTok{,}
    \DataTypeTok{linetype=}\DecValTok{0}
\NormalTok{  ) }\OperatorTok{+}
\StringTok{  }\KeywordTok{scale_y_continuous}\NormalTok{(}
    \DataTypeTok{name=}\StringTok{"Starting position coverage"}\NormalTok{,}
    \DataTypeTok{limits=}\KeywordTok{c}\NormalTok{(}\FloatTok{0.0}\NormalTok{, }\FloatTok{1.05}\NormalTok{)}
\NormalTok{  ) }\OperatorTok{+}
\StringTok{  }\KeywordTok{scale_x_continuous}\NormalTok{(}
    \DataTypeTok{name=}\StringTok{"Generation"}
\NormalTok{  ) }\OperatorTok{+}
\StringTok{  }\KeywordTok{scale_fill_brewer}\NormalTok{(}
    \DataTypeTok{name=}\StringTok{"Cardinaltiy"}\NormalTok{,}
    \DataTypeTok{palette=}\NormalTok{cb_palette}
\NormalTok{  ) }\OperatorTok{+}
\StringTok{  }\KeywordTok{scale_color_brewer}\NormalTok{(}
    \DataTypeTok{name=}\StringTok{"Cardinaltiy"}\NormalTok{,}
    \DataTypeTok{palette=}\NormalTok{cb_palette}
\NormalTok{  )}
\NormalTok{unique_start_positions_coverage_fig}
\end{Highlighting}
\end{Shaded}

\includegraphics{supplemental-material_files/figure-latex/unnamed-chunk-8-1.pdf}

\hypertarget{final-starting-position-coverage}{%
\subsection{Final starting position coverage}\label{final-starting-position-coverage}}

\begin{Shaded}
\begin{Highlighting}[]
\NormalTok{final_unique_start_positions_coverage_fig <-}\StringTok{ }\KeywordTok{ggplot}\NormalTok{(}
\NormalTok{    final_data,}
    \KeywordTok{aes}\NormalTok{(}
      \DataTypeTok{x=}\NormalTok{cardinality,}
      \DataTypeTok{y=}\NormalTok{unique_start_positions_coverage,}
      \DataTypeTok{fill=}\NormalTok{cardinality}
\NormalTok{    )}
\NormalTok{  ) }\OperatorTok{+}
\StringTok{  }\KeywordTok{geom_flat_violin}\NormalTok{(}
    \DataTypeTok{position =} \KeywordTok{position_nudge}\NormalTok{(}\DataTypeTok{x =} \FloatTok{.2}\NormalTok{, }\DataTypeTok{y =} \DecValTok{0}\NormalTok{),}
    \DataTypeTok{alpha =} \FloatTok{.8}\NormalTok{,}
    \DataTypeTok{scale=}\StringTok{"width"}
\NormalTok{  ) }\OperatorTok{+}
\StringTok{  }\KeywordTok{geom_point}\NormalTok{(}
    \DataTypeTok{mapping=}\KeywordTok{aes}\NormalTok{(}\DataTypeTok{color=}\NormalTok{cardinality),}
    \DataTypeTok{position =} \KeywordTok{position_jitter}\NormalTok{(}\DataTypeTok{width =} \FloatTok{.15}\NormalTok{),}
    \DataTypeTok{size =} \FloatTok{.5}\NormalTok{,}
    \DataTypeTok{alpha =} \FloatTok{0.8}
\NormalTok{  ) }\OperatorTok{+}
\StringTok{  }\KeywordTok{geom_boxplot}\NormalTok{(}
    \DataTypeTok{width =} \FloatTok{.1}\NormalTok{,}
    \DataTypeTok{outlier.shape =} \OtherTok{NA}\NormalTok{,}
    \DataTypeTok{alpha =} \FloatTok{0.5}
\NormalTok{  ) }\OperatorTok{+}
\StringTok{  }\KeywordTok{scale_y_continuous}\NormalTok{(}
    \DataTypeTok{name=}\StringTok{"Starting position coverage"}\NormalTok{,}
    \DataTypeTok{limits=}\KeywordTok{c}\NormalTok{(}\DecValTok{0}\NormalTok{, }\FloatTok{1.05}\NormalTok{)}
\NormalTok{  ) }\OperatorTok{+}
\StringTok{  }\KeywordTok{scale_x_discrete}\NormalTok{(}
    \DataTypeTok{name=}\StringTok{"Cardinality"}
\NormalTok{  ) }\OperatorTok{+}
\StringTok{  }\KeywordTok{scale_fill_brewer}\NormalTok{(}
    \DataTypeTok{name=}\StringTok{"Cardinaltiy"}\NormalTok{,}
    \DataTypeTok{palette=}\NormalTok{cb_palette}
\NormalTok{  ) }\OperatorTok{+}
\StringTok{  }\KeywordTok{scale_color_brewer}\NormalTok{(}
    \DataTypeTok{name=}\StringTok{"Cardinaltiy"}\NormalTok{,}
    \DataTypeTok{palette=}\NormalTok{cb_palette}
\NormalTok{  ) }\OperatorTok{+}
\StringTok{  }\KeywordTok{theme}\NormalTok{(}
    \DataTypeTok{legend.position=}\StringTok{"none"}
\NormalTok{  )}
\NormalTok{final_unique_start_positions_coverage_fig}
\end{Highlighting}
\end{Shaded}

\includegraphics{supplemental-material_files/figure-latex/unnamed-chunk-9-1.pdf}

\hypertarget{manuscript-figures}{%
\section{Manuscript figures}\label{manuscript-figures}}

Combine figures for the manuscript.

\begin{Shaded}
\begin{Highlighting}[]
\NormalTok{grid <-}\StringTok{ }\KeywordTok{plot_grid}\NormalTok{(}
\NormalTok{  elite_trait_ave_fit }\OperatorTok{+}
\StringTok{    }\KeywordTok{ggtitle}\NormalTok{(}\StringTok{"Performance over time"}\NormalTok{) }\OperatorTok{+}
\StringTok{    }\KeywordTok{theme}\NormalTok{(}\DataTypeTok{legend.position=}\StringTok{"none"}\NormalTok{),}
\NormalTok{  elite_trait_ave_fit_final }\OperatorTok{+}
\StringTok{    }\KeywordTok{ggtitle}\NormalTok{(}\StringTok{"Final performance"}\NormalTok{) }\OperatorTok{+}
\StringTok{    }\KeywordTok{theme}\NormalTok{(),}
\NormalTok{  unique_start_positions_coverage_fig }\OperatorTok{+}
\StringTok{    }\KeywordTok{ggtitle}\NormalTok{(}\StringTok{"Start position coverage over time"}\NormalTok{) }\OperatorTok{+}
\StringTok{    }\KeywordTok{guides}\NormalTok{(}\DataTypeTok{color=}\KeywordTok{guide_legend}\NormalTok{(}\DataTypeTok{nrow =} \DecValTok{1}\NormalTok{), }\DataTypeTok{fill=}\KeywordTok{guide_legend}\NormalTok{(}\DataTypeTok{nrow=}\DecValTok{1}\NormalTok{)) }\OperatorTok{+}
\StringTok{    }\KeywordTok{theme}\NormalTok{(}
      \DataTypeTok{legend.position=}\StringTok{"bottom"}\NormalTok{,}
      \DataTypeTok{legend.box=}\StringTok{"horizontal"}
\NormalTok{    ),}
\NormalTok{  final_unique_start_positions_coverage_fig }\OperatorTok{+}
\StringTok{    }\KeywordTok{ggtitle}\NormalTok{(}\StringTok{"Final start position coverage"}\NormalTok{) }\OperatorTok{+}
\StringTok{    }\KeywordTok{theme}\NormalTok{(),}
  \DataTypeTok{nrow=}\DecValTok{2}\NormalTok{,}
  \DataTypeTok{ncol=}\DecValTok{2}\NormalTok{,}
  \DataTypeTok{rel_widths=}\KeywordTok{c}\NormalTok{(}\DecValTok{2}\NormalTok{,}\DecValTok{1}\NormalTok{),}
  \DataTypeTok{labels=}\StringTok{"auto"}
\NormalTok{)}

\KeywordTok{save_plot}\NormalTok{(}
  \KeywordTok{paste}\NormalTok{(working_directory, }\StringTok{"imgs/cardinality-panel.pdf"}\NormalTok{, }\DataTypeTok{sep=}\StringTok{""}\NormalTok{),}
\NormalTok{  grid,}
  \DataTypeTok{base_width=}\DecValTok{12}\NormalTok{,}
  \DataTypeTok{base_height=}\DecValTok{10}
\NormalTok{)}

\NormalTok{grid}
\end{Highlighting}
\end{Shaded}

\includegraphics{supplemental-material_files/figure-latex/unnamed-chunk-10-1.pdf}

\bibliography{packages.bib,supplemental.bib}

\end{document}
